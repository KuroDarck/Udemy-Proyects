                *Orden Del contenido De Las Carpetas de ITCSS*

El modelo ITCSS (Inverted Triangle CSS) es una metodología de organización de estilos CSS 
que propone una estructura en forma de triángulo invertido, donde los estilos más generales 
y de alto nivel se encuentran en la parte superior, y los estilos más específicos 
y de bajo nivel se encuentran en la parte inferior.

Aquí hay algunos ejemplos de utilidades indispensables que podrías utilizar 
en cada una de las capas del modelo ITCSS:
                
    1. Settings:
                   - Variables globales: Define variables para colores, tamaños de fuente, espaciado, etc.
                   - Configuraciones de breakpoints: Define los puntos de quiebre para los diseños responsivos.
                
    2. Tools:
                   - Mixins: Crea mixins reutilizables para estilos comunes, como fuentes, animaciones, etc.
                   - Funciones: Utiliza funciones para realizar cálculos y manipulación de valores CSS.
                
    3. Generic:
                   - Normalize/Reset: Aplica estilos básicos de normalización o reseteo para garantizar una base consistente en todos los navegadores.
                   - Estilos base: Define estilos generales para elementos HTML básicos, como `body`, `p`, `h1`, etc.
                
    4. Elements:
                   - Estilos de etiquetas HTML: Aplica estilos específicos para elementos HTML, como enlaces (`a`), imágenes (`img`), listas (`ul`, `ol`), etc.
                   - Estilos de componentes básicos: Define estilos para componentes reutilizables básicos, como botones, formularios, etc.
                
    5. Objects:
                   - Grid: Define estilos para un sistema de grillas responsivo.
                   - Contenedores: Establece estilos para contenedores generales, como encabezados, pies de página, barras laterales, etc.
                
    6. Components:
                   - Estilos de componentes complejos: Define estilos para componentes más complejos, como carruseles, menús desplegables, etc.
                   - Estilos de módulos de interfaz de usuario: Aplica estilos para componentes de interfaz de usuario más pequeños, como cajas de diálogo, tarjetas, etc.
                
    7. Trumps:
                   - Clases utilitarias: Crea clases utilitarias para realizar ajustes rápidos y específicos en estilos existentes, como márgenes, rellenos, alineaciones, etc.
                   - Clases de estado: Define clases para aplicar estilos de estado, como `.active`, `.disabled`, etc.
                
Estos son solo ejemplos de utilidades que podrías utilizar en cada capa del modelo ITCSS. 
La idea es que los estilos se organicen de manera jerárquica, comenzando desde lo más general 
y avanzando hacia lo más específico, lo que facilita la legibilidad, el mantenimiento y 
la reutilización de los estilos en un proyecto web.